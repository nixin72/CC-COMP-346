% Created 2019-09-24 Tue 19:48
% Intended LaTeX compiler: pdflatex
\documentclass[11pt]{article}
\usepackage[utf8]{inputenc}
\usepackage[T1]{fontenc}
\usepackage{graphicx}
\usepackage{grffile}
\usepackage{longtable}
\usepackage{wrapfig}
\usepackage{rotating}
\usepackage[normalem]{ulem}
\usepackage{amsmath}
\usepackage{textcomp}
\usepackage{amssymb}
\usepackage{capt-of}
\usepackage{hyperref}
\date{\today}
\title{}
\hypersetup{
 pdfauthor={},
 pdftitle={},
 pdfkeywords={},
 pdfsubject={},
 pdfcreator={Emacs 26.2 (Org mode 9.2.4)}, 
 pdflang={English}}
\begin{document}

\tableofcontents

```
while (true) \{
  flag[i] = true
  turn = j
  while (flag[j] \&\& turn == j) \{
    // critical section
    flag[i] = false
    //remainder section
  \}
\}

```

\#\# Peterson's Solution
\#\# Synchronization Hardware

\begin{itemize}
\item Many sys provide HW support for implementing the critical section code.
\item Uniprocessors
\begin{itemize}
\item Code would execute wi/out preemption
\item Too inefficient on multi-processor sys
\begin{itemize}
\item OSs using this not scalable
\end{itemize}
\end{itemize}
\item Hardware support
\begin{itemize}
\item Memory barriers
\item Hardware instructions
\item Atomic variables
\end{itemize}
\end{itemize}

\begin{center}
\begin{tabular}{ll}
Hardware & SW + HW\\
:------------------- & :----------\\
Disable interrupts & mutex lock\\
Atomic instructions & semaphonre\\
-- test\textsubscript{and}\textsubscript{set} & monitor\\
-- compare\textsubscript{and}\textsubscript{swap} & \\
\end{tabular}
\end{center}

Users cannot block interrupts - must be kernel
No I/O inside critical section code - don't block anything. Can have a ContextSwitch though
Atomic: There may be a ContextSwitch, there may not be.

\#\# Memory Barriers
\begin{itemize}
\item An instruction that forces any change in memory to be propagated to all other processes
\item Memory model: mem guarantees a comp. arch. makes to application programs
\begin{itemize}
\item Strongly ordered: all mem mods of one processors is immediately visible to all other processes
\item Weakly ordered: all mem mods of one process may not be immediately visible to all other processes
\end{itemize}

\item Could add a mem barrier to the following to ensure thread 1 outputs 100:
\end{itemize}
t1 now does:
```
while(!flag):
  mem\textsubscript{barrier}()
print x
```

t2 now does:
```
x = 100
mem\textsubscript{barrier}()
flag = true
```

\#\# Hardware Instructions
Special HW instructions that \textbf{test-and-modify} the content of a word, or \textbf{swap} the contents of two words
\begin{itemize}
\item \textbf{\textbf{Test-and-set}} instruction
\item \textbf{\textbf{Compare and swap}} instruction
\end{itemize}

\#\#\# test\textsubscript{and}\textsubscript{set} instruction
```
bool test\textsubscript{and}\textsubscript{set}(boolean* target) \{
  bool rv = *target
  *target = true
  return rv
\}
```
\begin{itemize}
\item Executed atomically
\item Returns the og value of the passed param
\item Set the new val of passed param to true
\end{itemize}

\#\#\# compare\textsubscript{and}\textsubscript{swap}
```
int compare\textsubscript{and}\textsubscript{swap}(int* value, int expected, int new\textsubscript{val}) \{
  int temp = *value

  if (*value == expected)
    *value = new\textsubscript{value}
  return temp
\}
```
\begin{itemize}
\item exec atomicaaly
\item Return the og val of the passes param `val`
\item Set the var `val` the val of the passed param `new\textsubscript{val}`
\end{itemize}

\#\#\# Solution using test\textsubscript{and}\textsubscript{swap}
\begin{itemize}
\item Shared boolean variable `lock`, intialized to `false`
\item Solution:
\end{itemize}
```
do \{
  while test\textsubscript{and}\textsubscript{set}(\&lock)
    \emph{/ do nothing
      /} crictical section
    lock = false
      // remainder section
\} while true
```
Critical section problem:
\begin{itemize}
\item Satisfies Mutual exclusion problem
\item Satisfied Progress problem
\item Does NOT satisfy Bounded waiting problem
\begin{itemize}
\item One process can grab the same lock over and over again
\item Process can starve.
\item Will never enter critical section because it can never lock
\item BUT, there's no deadlock or livelock
\end{itemize}
\end{itemize}

\#\#\# Solution using compare\textsubscript{and}\textsubscript{swap}
Shared int `lock` initialized to 0
```
while true:
  while compare\textsubscript{and}\textsubscript{swap}(\&lock, 0, 1) != 0:
    //do nothing

  // critical section
  lock = 0
  //remainder section
```

\#\#\#\# Bounded waiting mutual-exclusion with compare\textsubscript{and}\textsubscript{swap}
```
while true:
  waiting[i] = true
  key = 1
  while waiting[i] and key == 1:
    key = compare\textsubscript{and}\textsubscript{swap}(\&lock, 0, 1)
  waiting[i] = false

//critical section

j = (i+1) \% n
while (j != i) and !waiting[j]:
  j = (j + 1) \% n

if j == i:
  lock = 0
else:
  waiting[j] = false

  //remainder section
```
\begin{itemize}
\item Solves critical section problem
\item Satisfies bounded waiting problem
\item Has to wait (at most) for n-1 processes to grab the lock. Checks processes in a round-robin way to grant lock.
\end{itemize}

\#\# Atomic Variables
\begin{itemize}
\item Typically, instructions such as `compare\textsubscript{and}\textsubscript{swap}` are used as building blocks for other synchronization tools
\item One tool is an atomic variable that provides atomic (uninteruptible) updates on basic data types such as ints and bools
\item ex: the `increment()` operation on the atomic var `sequence` ensures `sequence` is incremented without interruption
\end{itemize}
`increment(\&sequence)`

\#\#\# Consumer
```py
while true:
  while counter == 0
    // do nothing
  next\textsubscript{consumed} = buffer[out]
  out = (out + 1) \% BUFFER\textsubscript{SIZE}
  counter--
  //consume the item in next consumed
```

\#\#\# Race Conditions
\begin{itemize}
\item counter++
\end{itemize}
could be: `reg1 = count; reg1 = reg1 + 1; count = reg1`
\begin{itemize}
\item counter--
\end{itemize}
could be: `reg2 = count; reg2 = reg2 - 1; count = reg2`
\begin{itemize}
\item Consider:
\end{itemize}
s0 producer exec reg1 = counter
s1 producer exec reg1  

\begin{itemize}
\item The `increment()` function can be implemented as:
\end{itemize}
```python
void increment(atomic\textsubscript{int}* v):
  int temp
  do:
    temp = *v
  while temp != compare\textsubscript{and}\textsubscript{swap}(v, temp, temp+1)
```

\#\# Mutex Locks
\begin{itemize}
\item Previous solutions are complicated and generally inaccessbile to application programmers
\item OS designers build software tools to solve critical section problem
\item Simplest is Mutex Lock
\item Protect a C.S by first `acquire()` a lock, then `release()` the lock
\begin{itemize}
\item bool var indicating if lock is available or not
\end{itemize}
\item Call to `acquire()` and `releae()` must be atomic
\begin{itemize}
\item Usually implemeted via a hardware atomic instructions such as compare$\backslash$\textsubscript{and}\textsubscript{swap}
\end{itemize}
\item But this solution requires \textbf{\textbf{Busy waiting}}
\begin{itemize}
\item This lock is therefore called a \textbf{\textbf{spinlock}}
\end{itemize}
\end{itemize}

\#\#\# Definitions 
```
acquire()
  while !available
    // busy wait
  available = false
```

``` python
release()
  available = true
```

these two funcs must be implemeted atomicaaly. Both test\textsubscript{and}\textsubscript{set} and compare\textsubscript{and}\textsubscript{swap} can be used to implement these functions.

\#\# Semaphore
\begin{itemize}
\item Synchronization tool that provides more sophisticated ways (than mutex locks) for process to sync their activities
\item Semaphore S - integer var
\item Can only be accessed via two indivisible (atomic) operations 
\begin{itemize}
\item `wait()` and `signal()`
\begin{itemize}
\item (originally called P() and V())
\end{itemize}
\end{itemize}
\item Definition of `wait()` and `signal()` operations
\end{itemize}
``` python
wait(S) 
  while S <= 0
    // busy wait
  S--
```
``` python

signal(S):
  S++
```
\end{document}
